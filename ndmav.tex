\documentclass[10pt, t, xcolor={usenames,dvipsnames,svgnames}, compress]{beamer}

\usepackage{booktabs}
\usepackage{dcolumn}
\usepackage{colortbl}
\usepackage{ifxetex}
\usepackage{amsmath}
\usepackage[style=authoryear-icomp, maxcitenames=2]{biblatex}
\usepackage[no-math]{fontspec}

\definecolor{untractable_red}{RGB}{209, 25, 25}
\definecolor{tractable_green}{RGB}{0, 153, 51}

\usetheme{enziteto}

\setbeamertemplate{headline}{}

\newcommand{\argmax}{\operatornamewithlimits{argmax}}

\addbibresource{../referomnia/referomnia.bib}

\title{Learning Sum-Product Networks}
\author{Nicola Di Mauro \and Antonio Vergari}
\date{September 2016}
\institute{Università degli Studi di Bari}
\department{Dipartimento di Informatica}
\laboratory{LACAM}
\group{Machine Learning}
\institutelogo{\includegraphics[width=25pt]{figures/unibaba}}
\lablogo{\includegraphics[width=35pt]{figures/lacam}}

%\footnotesize \let\small\footnotesize

\setbeamerfont{footnote}{size=\scriptsize}
\addtobeamertemplate{footnote}{}{\vspace{16pt}}

\begin{document}

\begin{frame}[c]
  \setbeamertemplate{headline}{}
  \setbeamertemplate{footline}{}
  \titlepage
\end{frame}

\begin{frame}
\frametitle{The need for SPN}
  \framesubtitle{Why should you work on SPNs?}

Probabilistic modeling of data aims at
\begin{itemize}
  \item representing probability distributions \emph{compactly}
  \item computing their marginals and modes \emph{efficiently} (inference)
  \item learning them \emph{accurately}
\end{itemize}
A solution is to use Probabilistic Graphical Models (PGMs).
%$P(X) = \frac{1}{Z}\prod_k \phi_k(x_{\{k\}})$
%with $Z = \sum_x \prod_k \phi_k(x_{\{k\}})$.

However, PGMs are limited in
\begin{itemize}
\item representing compact distributions
\item having intractable exact inference in the worst case 
  \begin{itemize}
  \item falling back on approximate inference
  \end{itemize}
\item requiring and exponential sample size (wrt the number of variables)
\item learning the structure since it requires inference
\end{itemize}
\end{frame}

\begin{frame}
\frametitle{The need for SPN (II)}
  \framesubtitle{Why should you work on SPNs?}

Sum-Product Networks (SPNs) are a type of probabilistic model\footnote{H. Poon
  and P. Domingos, \emph{Sum-Product Network: a New Deep Architecture}, UAI 2011}
\begin{itemize}
\item a deep model with multiple layer of hidden variables
\item tractable exact inference 
\end{itemize}
SPNs represent probability distributions and a corresponding exact inference
machine for the represented distribution at the same time 
\end{frame}


\section{Representation}
{\setbeamertemplate{headline}{}
  \begin{frame}[c]
    \sectionpage
  \end{frame}
}

\begin{frame}
  \frametitle{Density estimation}
\end{frame}

\begin{frame}
  \frametitle{(Different kinds of) Inference}
  Different kinds of queries:
  \begin{itemize}
  \item $p(\mathbf{X})$ (evidence)
  \item $p(\mathbf{E}), \mathbf{E}\subset\mathbf{X}$ (marginals)
  \item $p(\mathbf{Q}|\mathbf{E}), \mathbf{Q},
    \mathbf{E}\subset\mathbf{X}, \mathbf{Q}\cap \mathbf{E}=\emptyset$ (conditionals)
  \item
    $\arg\max_{\mathbf{q}\sim\mathbf{Q}}p(\mathbf{q}|\mathbf{E})$
    (MPE assignment)
    \item complex queries
  \end{itemize}
\end{frame}

\begin{frame}
  \frametitle{Tractable Probabilistic Models}
\end{frame}

\begin{frame}
  \frametitle{Sum-Product Networks}
\end{frame}

\begin{frame}
  \frametitle{Scopes}
\end{frame}

\begin{frame}
  \frametitle{Structural Properties}
\end{frame}

\section{Inference}
{\setbeamertemplate{headline}{}
  \begin{frame}[c]
    \sectionpage
  \end{frame}
}

\begin{frame}
  \frametitle{Complete evidence}
\end{frame}

\begin{frame}
  \frametitle{Marginal inference}
\end{frame}

\begin{frame}
  \frametitle{MPE inference}
\end{frame}


\section{Interpretation}
{\setbeamertemplate{headline}{}
  \begin{frame}[c]
    \sectionpage
  \end{frame}
}

\begin{frame}
\frametitle{Interpretation}
\begin{itemize}
\item probabilistic model
\item deep feedforward neural network
\end{itemize}
\end{frame}


\begin{frame}
\frametitle{Network Polynomials}
\end{frame}

\begin{frame}
  \frametitle{Arithmetic Circuits}
  Differences with ACs:
  \begin{itemize}
  \item probabilistic semantics
    \begin{itemize}
    \item learning
      \item sampling
    \end{itemize}
    \item no shared weights
  \end{itemize}
\end{frame}

\begin{frame}[t]
  \frametitle{SPNs as NNs (I)}
  SPNs are a particular kind of \emph{\textbf{labelled}
    \textbf{constrained} and \textbf{fully probabilistic}}
  neural networks.\par\bigskip
  
  \textbf{Labelled}: each neuron is associated a \emph{scope}\par
  \textbf{Constrained}: completeness and decomposability determine
  network topology.\par
  \textbf{Fully probabilistic:} each valid sub-SPN is still a
  valid-SPN.\footfullcitenomarkleft{Vergari2016a}\par\bigskip
  
  SPNs provide a direct encoding of the input space into a deep
  architecture $\rightarrow$ \emph{\textbf{visualizing representations}} (back) into the \emph{\textbf{input space}}.
\end{frame}

\begin{frame}
  \frametitle{SPNs as NNs (II)}
  \small
  A classic MLP hidden layer computes the function:
  $$h(\mathbf{x}) =\sigma(\mathbf{W}\mathbf{x}+ \mathbf{b})$$

  SPNs can be reframed as \textit{DAGs} of MLPs, each sum layer
  computing:
  $$\mathbf{S}(\mathbf{x}) =
  \log(\mathbf{W}\mathbf{x})$$
  and product layers computing:
  $$\mathbf{S}(\mathbf{x}) = \exp(\mathbf{P}\mathbf{x})$$
  where
  $\mathbf{W}\in\mathbb{R}_{+}^{s\times r}$ and $\mathbf{P}\in\{0,
  1\}^{s\times r}$ are the weight matrices:
  \begin{equation*}
    \mathbf{W}_{(ij)}= \begin{cases}
      w_{ij} &\text{if $i\rightarrow j$}\\
      0& \text{otherwise}
    \end{cases}\quad\quad\mathbf{P}_{(ij)}=
    \begin{cases}
      1 &\text{if $i\rightarrow j$}\\
      0& \text{otherwise}
    \end{cases}
  \end{equation*}\footfullcitenomarkleft{Vergari2016a}
\end{frame}

\begin{frame}
  \frametitle{SPNs as NNs (III)}
  \begin{center}
    \includegraphics[width=0.8\columnwidth]{figures/layered-spn.pdf}
  \end{center}
\end{frame}

\begin{frame}
  \frametitle{SPNs as NNs (IV): filters}
  \small
  Learned features as images maximizing neuron activations~\parencite{Erhan2009}:
  $$\mathbf{x}^{*} = \argmax_{\mathbf{x},
    ||\mathbf{x}||=\gamma}h_{ij}(\mathbf{x};\boldsymbol\theta).$$
  With SPNs, joint solution as an MPE assignment for all nodes (linear time):
  $$\mathbf{x}^{*}_{|\mathsf{sc}(n)} =
  \argmax_{\mathbf{x}}S_{n}(\mathbf{x}_{|\mathsf{sc}(n)};
  \mathbf{w})$$.
  \begin{center}
    \vspace{-15pt}
    \includegraphics[width=0.9\linewidth]{figures/filters.jpg}
  \end{center}
  \footfullcitenomarkleft{Vergari2016a}
  $\rightarrow$ \emph{scope length} ($|\mathsf{sc}(n)|$) correlates with feature abstraction level
\end{frame}

\begin{frame}
  \frametitle{SPNs as BNs I}
  Zhao and Poupart
\end{frame}

\begin{frame}
  \frametitle{SPNs as BNs II}
  Peharz
\end{frame}

\begin{frame}
  \frametitle{Myths about SPNs}
 \textbf{SPNs are PGMs}. false

 \textbf{SPNs are convolutional NNs}. false

  SPNs 
\end{frame}

\section{Learning}
{\setbeamertemplate{headline}{}
  \begin{frame}[c]
    \sectionpage
  \end{frame}
}

\begin{frame}
  %\small
  \frametitle{Learning SPNs}

  \textbf{Parameter learning:} estimate $\mathbf{w}$ from data
  considering an SPN as a
  latent RV model, or as a NN\par\bigskip

  \textbf{Structure learning:} build the network from data by assigning
  scores to tentative structures or by exploiting constraints\par\bigskip

  How to learn a ``complete'' SPN:
  \begin{itemize}
  \item handcrafted structure, then parameter learning~\parencite{Poon2011,Gens2012}
  \item random structures, then parameter learning~\parencite{Rashwan2016}
  \item structure learning, then parameter learning (fine tuning)~\parencite{Zhao2016a}
  \item learn both weight and structure at the same time~\parencite{Gens2013,Rooshenas2014,Vergari2015,Adel2015}\dots  
  \end{itemize}
\end{frame}

\begin{frame}
  \frametitle{Structure Learning}

  \textbf{Score vs constraint based search.}
  No closed form for likelihood scores, need
  heuristics~\parencite{Rooshenas2014}.\par
  No need for it by exploiting the inner nodes probabilistic semantics\par\bigskip
  
  \textbf{Learning graph vs tree structures}:
  Easier to learn a tree SPN (sometimes SPT) with greedy
  approaches. Graph SPNs may be more compact and expressive efficient.\par\bigskip

  \textbf{Top-down vs bottom-up approaches}: iteratively cluster data
  matrix (top-down) or start by the marginal RVs (bottom-up)\par\bigskip

  \textbf{LearnSPN}~\parencite{Gens2013} is a \emph{\textbf{greedy}}, \emph{\textbf{top down}}, \emph{\textbf{constraint
  based}} learner for \emph{\textbf{tree}} SPNs\par
  $\rightarrow$ First principled top-down learner, inspired many algorithms and
  variations.
  $\rightarrow$ Surprisingly simple and accurate.
\end{frame}



\begin{frame}[t]
  \frametitle{LearnSPN (I)}
  Build a tree SPN by recursively split the data matrix:

  \begin{itemize}
  \item splitting columns into pairs by a greedy \textbf{\emph{G Test}}-based
    procedure with threshold $\rho$:
    \[
      G(X_i, X_j) =  2\sum_{x_i \sim X_i}\sum_{x_j \sim X_j}c(x_i, x_j)\cdot \log\frac{c(x_i, x_j)\cdot |T|}{c(x_i)c(x_j)}
    \]
  \item clustering instances into $|C|$ sets with \textbf{\emph{online Hard-EM}} with cluster penalty
    $\lambda$:
    \[\begin{array}{cc}
        Pr(\mathbf{X})= \sum_{C_i \in \mathbf{C}}\prod_{X_j \in \mathbf{X}}Pr(X_j|C_i)Pr(C_i)\\
        % & Pr(C_i) \propto e^{-\lambda |\mathbf{C}|\cdot |\mathbf{X}|}\\
      \end{array}\]
    weights are estimated as cluster proportions
  \item if there are less than $m$ instances, put a \textbf{\emph{naive
        factorization}} over leaves
  \item each univariate distribution get \emph{\textbf{ML estimation}} smoothed by $\alpha$  
  \end{itemize}\par\bigskip

  Hyperparameter space: $\{\rho, \lambda, m, \alpha\}$.
\end{frame}

\begin{frame}
  \frametitle{LearnSPN (II)}
  \footnotesize
  \onslide<1-4>{\begin{minipage}[t]{0.3\linewidth}
      \begin{center}
        \only<1>{\includegraphics[width=0.8\linewidth]{figures/grid-0}}
        \only<2-4>{\includegraphics[width=0.715\linewidth]{figures/grid-1}}
      \end{center}
    \end{minipage}}\hspace{10pt}\onslide<3-4>{\begin{minipage}[t]{0.3\linewidth}
      \begin{center}
        \includegraphics[width=0.72\linewidth]{figures/grid-2}
      \end{center}
    \end{minipage}}\hspace{10pt}\onslide<4>{\begin{minipage}[t]{0.3\linewidth}
      \begin{center}
        \includegraphics[width=0.735\linewidth]{figures/grid-3}
      \end{center}
    \end{minipage}}\\
  \vspace{15pt}
  \onslide<2-4>{\raisebox{0pt}{\begin{minipage}[t]{0.3\linewidth}
        \begin{center}
          \includegraphics[width=0.715\linewidth]{figures/learnspn-1}
        \end{center}
      \end{minipage}}}\hspace{5pt}\onslide<3-4>{\raisebox{-25pt}{\begin{minipage}[t]{0.3\linewidth}
        \begin{center}
          \includegraphics[width=0.83\linewidth]{figures/learnspn-2}
        \end{center}
      \end{minipage}}}\hspace{4pt}\onslide<4>{\raisebox{-22pt}{\begin{minipage}[t]{0.3\linewidth}
        \begin{center}
          \includegraphics[width=1.0\linewidth]{figures/learnspn-3}
        \end{center}
      \end{minipage}}}
\end{frame}

\begin{frame}
  \frametitle{Tweaking LearnSPN}
%  \footnotesize
  
  \textsf{LearnSPN} performs two interleaved greedy
  \textbf{\emph{hierarchical divisive clustering}}
  processes (co-clutering on the data matrix).\par\bigskip

  Fast and simple. But both processes never look back and are
  committed to the choices they take \emph{$\rightarrow$ slow down the two
  processes}\par\bigskip

  Online EM does not need to specify the number of clusters $k$ in
  advance. But overcomplex structures are learned by exploding the number of sum
  node children \emph{$\rightarrow$ look for deeper networks}\par\bigskip

  Tractable leaf estimation. But too strong naive factorization independence
  assumptions, hard to regularize \emph{$\rightarrow$ learn tree
    distributions as leaves}\par\bigskip

  ML estimations are effective. But they are not robust to noise, they
  can overfit the training set easily
  \emph{$\rightarrow$ learn bagged sum nodes}
\end{frame}

\begin{frame}
  \frametitle{Why Structure Quality Matters}

  Tractable inference is guaranteed \emph{if the network size is polynomial} in $|\mathbf{X}|$.\par\bigskip

  Network \emph{\textbf{size}} influences inference complexity: smaller networks,
  faster inference!\par
  $\rightarrow$ Comparing network sizes is better than comparing inference times\par\bigskip

  Network \emph{\textbf{depth}} influences \emph{expressive efficiency}~\emph{\parencite{Martens2014,Zhao2015}}.\par\bigskip

  Structural simplicity as a bias: overcomplex networks may not generalize well.\par\bigskip
  
  Structure quality desiderata: \textbf{\textbf{smaller}} but \textbf{\textbf{accurate}}, \textbf{\emph{deeper}} but not
  wider, SPNs.
\end{frame}

\begin{frame}
  \frametitle{LearnSPN-b}
  Observation: each clustering process benefits from the other one improvements/highly suffers
  from other's mistakes.\par\bigskip

  Idea: slow them down the processes by limiting the number of
  nodes to split to the minimum.
  \textsf{LearnSPN-b}, binary splitting $k=2$.\par%\bigskip

  $\rightarrow$ one hyperparameter less, $\lambda$.\par%\bigskip
  $\rightarrow$ not committing to complex structures too early\par
  $\rightarrow$ reducing node out fan increases the depth\par
  $\rightarrow$ same expressive power as LearnSPN structures\par
  $\rightarrow$ statistically same (or better) accuracy, smaller
  networks\par%\bigskip
  \begin{center}
    \raisebox{19pt}{\includegraphics[width=0.23\linewidth]{figures/learnspn-1.pdf}}
    \raisebox{30pt}{\hspace{20pt}\Huge$=$\hspace{10pt}}
    \includegraphics[width=0.3\linewidth]{figures/learnspn-4.pdf}
  \end{center}

  % \raisebox{45pt}{\begin{minipage}[t]{0.65\linewidth}
%       Objectives:
%       \begin{itemize}
%       \item not committing to complex structures too early
%       \item same expressive power as LearnSPN
%       \item reducing node out fan increases the depth
%       \item statistically same (or better) accuracy, smaller networks
%       \end{itemize}
%     \end{minipage}}\hspace{-5pt}\begin{minipage}[t]{0.3\linewidth}
%     \begin{center}
%       \includegraphics[width=0.9\linewidth]{figures/learnspn-4.pdf}
%     \end{center}
%   \end{minipage}\footfullcitenomark{Vergari2015}
%
  \end{frame}

\begin{frame}
  \frametitle{LearnSPN-b: depth VS size}
  \begin{figure}[htbp]
    \begin{center}
      \includegraphics[width=0.5\linewidth]{figures/nltcs-depth.pdf}
      \includegraphics[width=0.5\linewidth]{figures/plants-depth.pdf}
      \caption{\footnotesize
        Network sizes VS depths while varying the max
        number of sum node children splits ($k\in\{10, 4, 2\}$). Each dot is an experiment
        in the grid search hyperparameter space performed by
        \textsf{LearnSPN-b} on NLTCS (left) and Plants (right).}
    \end{center}
  \end{figure}
\end{frame}

\begin{frame}
  \frametitle{LearnSPN-b: best ll VS size}

  \begin{figure}[htbp]
    \begin{center}
      \includegraphics[width=0.65\linewidth]{figures/edges-comp.pdf}
      \caption{\footnotesize
        Comparing network sizes
        for the networks scoring the best log-likelihoods in the grid
        search as obtained by \textsf{LearnSPN}, \textsf{LearnSPN-b} and
        \textsf{LearnSPN-bT} for each dataset.}
    \end{center}
  \end{figure}
\end{frame}

\begin{frame}
  \frametitle{Other variations on LearnSPN}
  \textbf{ACs modeling leaves} by performing a greedy score
  search.
  \emph{\textbf{ID-SPN}} best log-likelihood learner (but lots of
  hyperparameters).\par
  Freely available in the \textsf{Libra}\footnote{http://libra.cs.uoregon.edu/} toolkit.~\parencite{Rooshenas2014}\par\bigskip
  
  Looking for \textbf{correlations instead of independencies} via matrix
  factorizations.\par
  Splitting matrix rows and columns at the same time: SPN-SVD.\par
  It can cope with continuous data~\parencite{Adel2015}\par\bigskip
  
  Post-learning \textbf{mergining sub-SPNs} that model ``similar'' distributions.\par
  Reducing network sizes~\parencite{Rahman2016}.\par\bigskip

  Learning Relational SPNs on \textbf{first order data}
  represented in Tractable Markov Logic (TML), LearnRSPN~\parencite{Nath2015}.
\end{frame}

\begin{frame}
  \frametitle{Other Tendencies in Structure Learning}
  \textbf{Learning deterministic structures} which enable closed form
  log-likelihood and weight estimation.\par
  Selective SPNs, enabling efficient Stochastic Local
  Search~\parencite{Peharz2014b}.\par
  Mixing latent and deterministic mixtures as sum
  nodes (a Cutset Network is an
  SPN!).~\parencite{Rahman2016}\par\bigskip
  
  \textbf{Learning DAGs} structures instead of trees.\par
  Substituting sub-structures with more complex ones by cloning mixtures.~\parencite{Dennis2015}\par\bigskip
  
  \textbf{Template learning} for sequence
  models.
  Stochastic local search over well defined constrained structures.
  Dynamic SPNs~\parencite{Melibari2016c}\emph{$\rightarrow$ PGM'16!}\par\bigskip
\end{frame}

\begin{frame}
  \frametitle{Parameter Learning}
  Non convex optimization, solvable with iterative methods like: GD,
  EM,\dots and their online variants\par\bigskip

  Use backpropagation ($\rightarrow$ differential approach) to compute
  $\nabla_{\mathbf{w}} S(\mathbf{x})$\par\bigskip

  Hard/Soft gradients\par\bigskip

  Some issues:
  \begin{itemize}
  \item vanishing gradients
    \item hyperparameter choices
  \end{itemize}
\end{frame}

\begin{frame}
  \frametitle{``Hard'' gradients}
  
\end{frame}

\begin{frame}
  \frametitle{Hard/Soft Parameter Learning}
  \begin{table}
    \centering
    \begin{tabular}{l c c}
      &\textbf{soft}& \textbf{hard}\\
      \textsf{Generative Gradient} &$\frac{\partial
                                     S(\mathbf{x})}{\partial w_{pc}} =
                                     S_{c}(\mathbf{x})\frac{\partial
                                     S(\mathbf{x})}{\partial S_{p}(\mathbf{x})}$
                    &
                      $\frac{\sharp\{w_{pc}\in
                      \mathsf{MPE-path}(\mathbf{x})\}}{w_{pc}}$\\
      \textsf{Discriminative Gradient} & $\frac{1}{S(\mathbf{y}|\mathbf{x})}\frac{\partial
                                         S(\mathbf{y}|\mathbf{x})}{\partial
                                         w_{pc}} - \frac{1}{S(\mathbf{*}|\mathbf{x})}\frac{\partial
                                         S(\mathbf{*}|\mathbf{x})}{\partial
                                         w_{pc}}$ &
                                                    $\frac{\sharp\{w_{pc}\in
                                                    \mathsf{MPE-path}(\mathbf{y}|\mathbf{x})\}
                                                    - \sharp\sharp\{w_{pc}\in
                                                    \mathsf{MPE-path}(\mathbf{1}|\mathbf{x})\}}{w_{pc}}$\\
      \textsf{Generative Posterior} & $p(H_{p}=c|\mathbf{x})\propto \frac{1}{S(\mathbf{x})}\frac{\partial
                                      S(\mathbf{x})}{\partial S_{p}(\mathbf{x})}S_{c}(\mathbf{x})w_{pc}$ & $p(H_{p}=c|\mathbf{x})=
                                                                                             \begin{cases}
                                                                                               1
                                                                                               \text{if}
                                                                                               w
                                                                                               \in W
                                                                                               \\
                                                                                               0 \text{otherwise}
                                                                                             \end{cases}
$\\
    \end{tabular}
  \end{table}
\end{frame}

\begin{frame}
  \frametitle{Bayesian Parameter Learning}
\end{frame}

\begin{frame}[t]
  \frametitle{Parameter Learning VS LearnSPN}
  \begin{table}
    \centering
    \scriptsize
    \setlength{\tabcolsep}{3pt}  
    \begin{tabular}{l r r r r r r r}
      \toprule
      & \textsf{LearnSPN}\footcite{Gens2013} & \textsf{LearnSPN-b}\footcite{Vergari2015} & \textsf{CVB-SPN}\footcite{Zhao2016a} & \textsf{OBMM}\footcite{Rashwan2016} & \textsf{SGD}\footcite{Rashwan2016} & \textsf{EM}\footcite{Rashwan2016} & \textsf{SEG}\footcite{Rashwan2016} \\
      \midrule
      \textbf{NLTCS} & -6.11 & -6.05 & -6.08 & -6.07 & -8.76 & -6.31 & -6.85\\
      \textbf{MSNBC} & -6.11& -6.04& -6.29& -6.03& -6.81 & -6.64 & -6.74 \\
      \textbf{KDDCup2k} & -2.18& -2.14& -2.14& -2.14& 44.53 & -2.20 & -2.34\\
      \textbf{Plants} & -12.98& -12.81& -12.86& -15.14& -21.50 & -17.68 & -33.47\\
      \textbf{Audio} & -40.50& -40.57& -40.36& -40.70& -49.35 & -42.55 & -46.31\\
      \textbf{Jester} & -53.48& -53.53& -54.26& -53.86& 63.89 & -54.26 & -59.48\\
      \textbf{Netflix} & -57.33& -57.73& -60.69& -57.99& 64.27 & -59.35 & -64.48\\
      \textbf{Accidents} & -30.04& -29.34& -29.55& -42.66& 53.69 & -43.54 & -45.59\\
      \textbf{Retail} & -11.04& -10.94& -10.91& -11.42& -97.11 & -11.42 & -14.94\\
      \textbf{Pumsb-star} & -24.78& -23.31& -25.93& -45.27& -128.48 & -46.54 & -51.84\\
      \textbf{DNA} & -82.52& -81.91& -86.73& -99.61& -100.70 & -100.10 & -105.25 \\
      \textbf{Kosarek} & -10.99& -10.72& -10.70& -11.22&34.64 & -11.87 & -17.71 \\
      \textbf{MSWeb} & -10.25& -9.83& -9.89& -11.33& -59.63 & -11.36 & -20.69\\
      \textbf{Book} & -35.89& -34.30& -34.44& -35.55& -249.28 & -36.13 & -42.95\\
      \textbf{EachMovie} & -52.49& -51.36& -52.63& -59.50& -227.05 & -64.76 & -84.82\\
      \textbf{WebKB} & -158.20& -154.28& -161.46& -165.57& -338.01 & -169.64 & -179.34\\
      \textbf{Reuters-52} & -85.07& -83.34& -85.45& -108.01& -407.96& -108.10 & -108.42\\
      \textbf{20-Newsgrp} & -155.93& -152.85& -155.61& -158.01& -312.12 & -160.41 & -167.89\\
      \textbf{BBC} & -250.69& -247.30& -251.23& -275.43& -462.96 & -274.82 & -276.97\\
      \textbf{Ad} & -19.73& -16.23& -19.00& -63.81& -638.43 & -63.83 & -64.11\\
      \bottomrule
    \end{tabular}
    \label{tab:model-accs}
  \end{table}

\end{frame}

\begin{frame}
  \frametitle{Parameter Learning}
  \framesubtitle{Why learning parameters}
  
\end{frame}

\section{Representation Learning}
{\setbeamertemplate{headline}{}
  \begin{frame}[c]
    \sectionpage
  \end{frame}
}

\begin{frame}
  \frametitle{Extracting Embeddings}
  \framesubtitle{Exploiting SPNs as feature extractors}

  Given an SPN $S$, generate a dense vector
  $$\mathbf{e}^{i} =f_{S}(\mathbf{x}^{i})$$ 
  
  Issues with SPNs as NNs:
  \begin{itemize}
  \item layer-wise extraction may be arbitrary
  \item power law distribution of nodes by scopes
    \item scope lengths as proxy for feature abstraction levels (see filter visualizations)
  \end{itemize}
  
  How to extract embeddings?
  \begin{itemize}
  \item from node outputs (one query) 
    \begin{itemize}
    \item all (non-leaf) nodes
    \item sum/product nodes only
    \item with a certain scope length
      \item aggregating node outputs by scope
    \end{itemize}
  \item from several queries evaluations (root outputs)
  \end{itemize}
\end{frame}

\begin{frame}
  \frametitle{Supervised classification}
  \framesubtitle{Experimental settings}
\end{frame}

\begin{frame}
  \frametitle{Embeddings (I)}
  \begin{table}[!t]
    \centering
    \small
    \caption[datasets]{Test set accuracy scores for the embeddings
      extracted with the best \textsf{SPN},
      \textsf{RBM} models and with the baseline \textsf{LR} model on all
      datasets. Bold values denote significantly better scores
      than all the others for a dataset.}
    \setlength{\tabcolsep}{3pt}  
    \begin{tabular}{l c c c c c c c}
      \toprule
      
      & \textsf{LR} & \textsf{SPN-I} & \textsf{SPN-II} & \textsf{SPN-III} & \textsf{RBM-5h} & \textsf{RBM-1k} & \textsf{RBM-5k} \\
      % &\emph{+ L2 reg}  & 500& 100& 50& 500& 1000& 5000\\
      % &  & ($d=227$)& ($d=5977$)& ($d=12669$)& & & \\
      \midrule
      \textsf{REC} & 69.28 & 77.31& \textbf{97.77}& 97.66& 94.22& 96.10& 96.36\\
      \midrule
      \textsf{CON} & 53.48 & 67.48& 78.31& \textbf{84.69}& 67.55& 75.37& 79.15\\
      \midrule
      \textsf{OCR} & 75.58 & 82.60& \textbf{89.95}& \textbf{89.94}& 86.07& 87.96& 88.76\\
      \midrule
      \textsf{CAL} & 62.67& 59.17 & 65.19& 66.62& 67.36& \textbf{68.88}& 67.71\\
      \midrule
      \textsf{BMN} & 90.62& 95.15& \textbf{97.66}& 97.59& 96.09& 96.80& 97.47\\
      \bottomrule
    \end{tabular}
    \label{tab:model-accs}
  \end{table}
\end{frame}

\begin{frame}
  \frametitle{Embeddings (II)}
  \begin{table}[!t]
    \caption[datasets]{Test set accuracy scores for the embeddings
      extracted with \textsf{SPN} models and filtered by node
      type. Results for \textsf{SPN-III} embeddings filtered by \textsf{S}mall
      , \textsf{M}edium and \textsf{L}arge scope
      lengths are reported in columns 8-10.
      Bold values denote significantly better scores than all the others.
      $\blacktriangle$ indicates a better score than an RBM embedding
      with greater or equal size. $\triangledown$ indicates worse
      scores than an RBM embedding with smaller or equal size.}
    \centering
    \small
    \setlength{\tabcolsep}{3pt}  
    \begin{tabular}{l c c c c c c | c c c}
      \toprule
      & \multicolumn{2}{c}{\textsf{SPN-I}} &
                                             \multicolumn{2}{c}{\textsf{SPN-II}}
      & \multicolumn{2}{c}{\textsf{SPN-III}} & \multicolumn{3}{c}{\textsf{SPN-III}}\\
      
      &sum  & prod& sum& prod& sum& prod & \textsf{S} & \textsf{M} & \textsf{L}\\
      % &  & (227)& (5977)& (12669)& & \\
      \midrule
      \textsf{REC} & 72.46& 62.25& $\mathbf{98.03}^{\blacktriangle}$& $97.06^{\blacktriangle}$& $\mathbf{98.00}^{\blacktriangle}$& $97.04^{\blacktriangle}$ & 88.73&$\mathbf{98.45}^{\blacktriangle}$& 93.91\\
      \midrule
      \textsf{CON} & 62.36& 64.03& $77.13^{\blacktriangle}$& $76.07^{\blacktriangle}$& $\mathbf{83.59}^{\blacktriangle}$& $82.06^{\blacktriangle}$ &$70.51^{\triangledown}$&77.18&$\mathbf{83.32}^{\blacktriangle}$\\
      \midrule
      \textsf{OCR} & 74.19& 81.58& $89.73^{\blacktriangle}$& $88.78^{\blacktriangle}$& $\mathbf{90.02}^{\blacktriangle}$& 89.32 & $87.22^{\triangledown}$& $\mathbf{89.29}^{\blacktriangle}$& $88.19^{\blacktriangle}$\\
      % & (32) & (621)&  (5466)& & & \\
      \midrule
      \textsf{CAL} & 38.19& 56.95& 62.64& 64.80& $\mathbf{66.58}^{\triangledown}$& $66.40^{\triangledown}$& $63.37^{\triangledown}$& $\mathbf{66.23}^{\triangledown}$& $66.10$\\
      % & (554 )& (16483)& (31420)& & & \\
      \midrule
      \textsf{BMN} & 93.50& 94.75& $97.67$& $96.90^{\triangledown}$& \textbf{97.80}& $97.20^{\triangledown}$ & $96.02^{\triangledown}$& $\mathbf{97.42^{\triangledown}}$& 97.38\\
      \bottomrule
    \end{tabular}
    \label{tab:model-filter-accs}
  \end{table}
\end{frame}

\begin{frame}
  \frametitle{Embeddings (III)}
  \begin{table}[!t]
    \caption[datasets]{Test set accuracy scores for the embeddings
      extracted with \textsf{SPN} models by aggregating node outputs
      with the same scope.
      Results for when leaves are not counted in the aggregarion
      (\textsf{no-leaves} columns) are reported alongside the case in which they are
      considered (\textsf{leaves} columns).
      Bold values denote significantly better scores than all the others
      for each dataset.
      $\blacktriangle$ indicates a better score than an RBM embedding
      with greater or equal size. $\triangledown$ indicates worse
      scores than an RBM embedding with smaller or equal size.}
    \centering
    \small
    \setlength{\tabcolsep}{3pt}  
    \begin{tabular}{l c c c c c c}
      \toprule
      & \multicolumn{2}{c}{\textsf{SPN-I}} &
                                             \multicolumn{2}{c}{\textsf{SPN-II}}
      & \multicolumn{2}{c}{\textsf{SPN-III}}\\
      
      &no-leaves  & leaves& no-leaves& leaves& no-leaves& leaves\\
      \midrule
      \textsf{REC} & 72.47 & $75.92^{\triangledown}$ & $\mathbf{97.94}^{\blacktriangle}$ & $\mathbf{97.99}^{\blacktriangle}$ & $\mathbf{97.94}^{\blacktriangle}$ & $\mathbf{98.02}^{\blacktriangle}$ \\
      \midrule
      \textsf{CON} & 62.35 & $66.49^{\triangledown}$ & $77.21^{\blacktriangle}$ & 78.05 & $\mathbf{83.52}^{\blacktriangle}$ & $\mathbf{83.84}^{\blacktriangle}$ \\
      \midrule
      \textsf{OCR} & 74.32 & 81.85 & $89.71^{\blacktriangle}$ & $89.68^{\blacktriangle}$ & $\mathbf{89.90}^{\blacktriangle}$ & $\mathbf{89.91}^{\blacktriangle}$ \\
      \midrule
      \textsf{CAL} & 38.10 & $63.19^{\triangledown}$  & 62.59 & $62.76^{\triangledown}$ & $\mathbf{66.49}^{\triangledown}$ & $\mathbf{66.58}^{\triangledown}$ \\
      \midrule
      \textsf{BMN} & 93.51 & $94.83^{\triangledown}$ & $97.64^{\blacktriangle}$ & $97.62^{\blacktriangle}$ & $\mathbf{97.80}$ & $\mathbf{97.80}$ \\
      \bottomrule
    \end{tabular}
    \label{tab:model-scope-aggr}
  \end{table}
\end{frame}



\begin{frame}
  \frametitle{Random Marginal Queries}
  Generate embeddings by asking several random queries to a black box
  density estimator.\par
  Eg. marginals: $e^{i}_{j}=P_{\theta}(\mathbf{Q}_{j}=\mathbf{x}^{i}_{\mathbf{Q}_{j}})$,
  according to estimator $\theta$ where $\mathbf{Q}_{j}\subseteq\mathbf{X}, j=\,\dots,k$.
  \begin{center}
    \includegraphics[width=1.0\linewidth]{figures/lines-wide}
  \end{center}
  \footfullcitenomarkleft{Vergari2016a}
  % Two general extraction schemes: random query construction and random
  % `patch' estimation
\end{frame}

\begin{frame}
  \frametitle{Encoding/Decoding Embeddings}
  MPN as autoencoders\footnote{Vergari et al. Encoding and Decoding
    Representations with Sum-Product Networks, 2016, to appear}.
\end{frame}




\section{Applications}
{\setbeamertemplate{headline}{}
  \begin{frame}[c]
    \sectionpage
  \end{frame}
}

\begin{frame}
  \frametitle{Applications I: computer vision}
  
\end{frame}


\begin{frame}
  \frametitle{Applications II: language modeling}
  
\end{frame}

\begin{frame}
  \frametitle{Applications III: activity recognition}
\end{frame}

\begin{frame}
  \frametitle{Applications IV: speech}
  SPNs to model the joint pdf of observed RVs in HMMs (HMM-SPNs).
  \begin{center}
    \includegraphics[width=0.35\columnwidth]{figures/peharz2014a-figures/hmm-spn-model}
  \end{center}\par\bigskip

  State-of-the-art high frequency reconstruction (MPE inference)
  \begin{center}
    \includegraphics[width=0.25\columnwidth]{figures/peharz2014a-figures/orig}
    \includegraphics[width=0.25\columnwidth]{figures/peharz2014a-figures/hmm-lp}
    \includegraphics[width=0.25\columnwidth]{figures/peharz2014a-figures/hmm-gmm}
    \includegraphics[width=0.25\columnwidth]{figures/peharz2014a-figures/hmm-spn}
  \end{center}
  
  \footfullcitenomarkleft{Peharz2014a}
\end{frame}


\begin{frame}
  \frametitle{Trends \& What to do next}
  Scalable structure learning
  Continuous RVs structure learning

  End-to-end training with hybrid NN architectures
\end{frame}

\section{References}
{\setbeamertemplate{headline}{}
  \begin{frame}[c]
    \sectionpage
  \end{frame}
}

\begin{frame}
  \frametitle{awesome-spn}
  A curated and structured list of resources about SPNs\footnote{Inspired by the
    SPN page {http://spn.cs.washington.edu/} at the Washington  University}.

  \url{https://github.com/arranger1044/awesome-spn}
\end{frame}

%\begin{frame} [allowframebreaks]
%  \setbeamertemplate{bibliography item}{}
%  \setlength\bibitemsep{8pt}
%  \printbibliography
%\end{frame}




\end{document}


%%% Local Variables:
%%% mode: latex
%%% TeX-master: t
%%% TeX-engine: xetex
%%% End:
